\documentclass{article}

\usepackage{fancyvrb}
\usepackage[T1]{fontenc}
\usepackage[utf8]{inputenc}
\usepackage[table]{xcolor}
\usepackage{framed,color,verbatim}
\usepackage{minted}
\usepackage{color}


\definecolor{codegray}{gray}{0.95}
\newcommand{\code}[1]{\colorbox{codegray}{\texttt{#1}}}

\newcommand{\bs}[1]{{\texttt{\textbackslash #1}}}

\title{Obligatorisk innlevering 1 høsten 2014, INF3331}
\author{Nicolai Skogheim <nicolai.skogheim@gmail.com>}
\date{\today}


\begin{document}
\maketitle

\section*{Oppgave 1.1}
Løsning (essensen):
\begin{minted}{sh}
folder=$1
days=$2

find -$folder -type f -mtime -${days}d | xargs du -h | sort -h}
\end{minted}


Find har flaggene \code{-type} og \code{-mtime}, for å velge hendholdsvis
filer (\code{-type f}) og treff som har mtime innenfor de siste n dagene.\\

\code{xargs} redistribuerer argumentene som kommer fra find gjennom pipa,
og sender dem til \code{du} som har \code{-h} (human readable) flagget for
å vise størrelsen på formen 1kb i stedet for 1024.\\

Til sist sorteres alt via \code{sort}-kommandoen, som sorterer etter størrelse
i human readable-format pga \code{-h}-flagget.\\

Utover det som er svaret på oppgaven har jeg gjort et par ting som ikke
var krevd, men som er god skikk i bash-programmer som krever argumenter.\\

Det første er å sjekke at hvis det ikke er sendt med to argumenter (\code{\$\#} = antall argumenter),
så skal det vises en liten melding med hjelp til brukeren.\\

Jeg valgte også legge input-parameterene i variabler bare for å få navn på dem.

\subsubsection*{Kjøring}
\begin{Verbatim}
14:36 $ bash list_new_files.sh file_tree 80
644K	file_tree/Kq0Wv/H/XhdhBbk
772K	file_tree/zg/Hu/vv/2KKnyIt5
780K	file_tree/Kq0Wv/MH/Z9kP8NB
780K	file_tree/Pkvye/vlfN/ZLbGhCmj
788K	file_tree/Kq0Wv/MH/zWG/8puxfjS
\end{Verbatim}


\section*{Oppgave 1.2}

Løsning: \\
\begin{minted}{sh}
find $1 -type f | xargs grep -n --color=always $2 \
|| echo No files containing \"$2\" found.
\end{minted}

find finner filer (samme som oppagaven over med "-type f"-flagget),
og sender treffene separert med linjeskift til stdout.
Pipen fanger stdout og peker den til stdin for xargs, som i sin tur
plasserer treffet i \code{grep} sin stdin. \code{grep}-flaggene
\code{-n} og \code{--color=always} gjør at grep hendholdsvis viser
linjenummer og farger treffene slik som vist i kjøreeksempelet.

De første to strekene i linje to er det man forventer fra andre språk,
altså en \code{OR}. Den siste linja vil altså bli kjørt bare hvis \code{grep}
ikke får noen treff, fordi \code{grep} da vil ha en returkode større enn 0.
Alle "gode" bash-programmer returnerer en statuskode som indikerer om operasjonen
gikk bra. I \code{grep} sitt tilfelle, er "bra" at man får treff, og da vil \code{grep}
returnere 0. Hvis det skjer, vil ikke den siste linja kjøre, fordi vi fikk treff.

\subsubsection*{Kjøring}
\begin{Verbatim}[commandchars=\\\{\}]
11:50 $ bash find_word.sh file_tree bil
file_tree/_CVcim:2006:eW68EPmXR\textcolor{red}{bil}ACbpN
file_tree/fJgme5F:2971:Hw\textcolor{red}{bil}R0c7PtSZ7fiUdc80q6jf3DIbS9_Kq9fe
file_tree/Kq0Wv/MH/7GvTL2y:4284:F\textcolor{red}{bil}umBAFScZqD3ih0_
file_tree/Kq0Wv/MH/_Oj2c0QA:7674:suJ19WkYf4_juYYVu4F\textcolor{red}{bil}L

11:51 $ bash find_word.sh file_tree java
No files containing "java" found.
\end{Verbatim}

\section*{Oppgave 1.3}

Løsning:
\begin{minted}{sh}
path=$1
size=$2

files_to_delete=$(find $path -size +${size}k -print)

if [ -n "$files_to_delete" ]
then
  echo Deleting...
  echo "$files_to_delete"
  rm -- $files_to_delete
else
  echo No files of size $size kilobytes or larger found.
fi

exit 0
\end{minted}

Først fanger jeg output fra \code{find} i variabelen \code{files\_to\_delete}
ved hjelp av \textit{variabelsubstitusjon}.
\code{-size} begrenser søket til en gitt størrelse, plussen betyr \textit{større enn},
og \textit{k} er for kilobyte.
Så blir det: \textit{hvis} \code{-n}(notempty) \code{\$files\_to\_delete}
så: output "Deleting", filer som skal slettes, og kjør kommandoen. \\
ellers: output "No files [...]".

Med \code{-\--} etter en kommando (eller \code{rm} i eksempelet) sier man at det ikke kommer flere flagg,
som i dette tilfellet betyr at programmet ikke tryner på filer som \textit{-testfil.txt}.

\subsubsection*{Kjøring}
\begin{Verbatim}
16:59 $ bash sized_delete.sh file_tree 80
Deleting...
file_tree/_CVcim
file_tree/fJgme5F
file_tree/Kq0Wv/MH/7GvTL2y
file_tree/Kq0Wv/MH/_Oj2c0QA
file_tree/Kq0Wv/MH/gBwNRP
file_tree/Kq0Wv/MH/XhdhBbk
file_tree/Kq0Wv/MH/Z9kP8NB


17:00 $ bash sized_delete.sh file_tree 80
No files of size 80 kilobytes or larger found.
\end{Verbatim}


\section*{Oppgave 1.4}

Løsning:
\begin{minted}{sh}
cat $1 | sort > $2
\end{minted}

\code{cat} leser fra førte argument (\$1), spytter linjene til sort,
som etter sortering sender output til andre argument (\$2) fordi det
er en \code{>} i mellom.\\

Dette tryner selvfølgelig hvis argumentene ikke er filer eller fildescriptorer.

\subsubsection*{Kjøring}

\begin{Verbatim}
17:09 $ bash sort_file.sh unsorted_fruits sorted_fruits

17:37 $ cat sorted_fruits
apple
grape
orange
pear
pineapple
\end{Verbatim}


\subsection*{Kommentar}
I noen av filene står det \code{\#!/usr/local/env bash} i stedet
for det vanlige \code{\#!/usr/bin}. Dette er fordi jeg jobber
på mac som kjører bash3.2 og mangler de kule gnu-verktøya, og derfor
trenger jeg å hente ting fra andre steder enn /bin/bash.
Sånn som det står nå vil det virke hos andre også.


\section*{Oppgave 2}


\subsubsection*{Kjøring}
En prøvekjøring kan gjøres ved å kjøre kommandoen
\begin{minted}{sh}
cd python
python my_generate_file_tree.py file_tree 10 10 --rec-depth 4
\end{minted}

\subsubsection*{Argumentparsing}
I stedet for metoden som var foreslått i startfila har jeg brukt
argparse til å holde styr på argumenter.\\

Skriv
\begin{minted}{sh}
python my_generate_file_tree -h
\end{minted}
for å få informasjon om hvordan man setter de forskjellige argumentene.


\subsubsection*{Funksjoner}
  Jeg har brukt \code{os.walk} i stedet for den utdaterte \code{os.path.walk}
  Den genererer en tuple du kan bruke i en loop, i stedet for at du sender
  med callback.

\subsubsection*{Randomness}
  Programmet vil stemple filer med tilfeldige atime's og mtime's,
  generere fra null til <grenseverdi> antall filer og mapper.
  Bortsett fra det er det random\_string som har ansvaret for
  tilfeldig lengde på fil- og mappenavn samt filinnhold.

\subsubsection*{Config-objektet}
  I og med at argumenter som sendes inn via terminalen gjelder
  hele app'en og det meste av funksjoner avhenger av dette
  har jeg valgt å legge alle argumentene i et config-objekt.\\

  Jeg mener det gjør koden enklere å ha med å gjøre, og jeg slipper
  de lange stygge signaturene med mange parametere og default-verdier.
  I tillegg havner plutselig dokumentasjonen for alt på samme sted,
  og det er ålreit.\\



\subsubsection*{Testing}
  Jeg har valgt å skrive tester, og disse kan kjøres med for eksempel
  \code{py.test -vls}


\subsubsection*{Kjøring av tester}
\begin{Verbatim}[commandchars=\\\{\}]
14:44 $ py.test -l
================================= test session starts ==================================
platform darwin -- Python 2.7.5 -- py-1.4.24 -- pytest-2.6.2
collected 12 items

generate_populate_tree_test.py ......
random_string_test.py ......

\color{green}============================== 12 passed in 1.12 seconds ===============================
\end{Verbatim}

\subsubsection*{Kjøring av program}
\begin{Verbatim}[commandchars=\\\{\}]
11:51 $ python my_generate_file_tree.py file_tree 5 5 --rec-depth 3 -v yes --seed 7653
\color{lime}Creating folder file_tree
\color{lime}Creating folder file_tree/IBYM
\color{lime}Creating folder file_tree/IBYM/_C
\color{lime}Creating folder file_tree/IBYM/_C/FZ
\color{cyan}Creating file file_tree/AaRr6
\color{cyan}Creating file file_tree/ip40OeguE
\color{cyan}Creating file file_tree/eSEzkLN
\color{cyan}Creating file file_tree/v
\color{cyan}Creating file file_tree/7nsit70RP/l

\color{yellow}Created 27 folders and 45 files, for a total of 14893 kilobytes
* Forkortet output
\end{Verbatim}


\end{document}
