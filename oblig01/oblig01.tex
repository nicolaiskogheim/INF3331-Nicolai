\documentclass{article}

\usepackage[T1]{fontenc}
\usepackage[utf8]{inputenc}
\usepackage{fancyvrb}

\author{
\{Jonathan Feinberg, Joakim Sundnes\} \\
\tt{\{jonathf,sundnes\}@simula.no}
}
\date{\today}
\title{1. obligatoriske innlevering, høsten 2014}

\begin{document}

\maketitle

\section*{Innleveringskrav}

Det forventes at alle skriptene beskrevet i oppgavene blir skrevet.
I tillegg skal en oppsummering skrives i et latex-dokument.
Denne skal følge malen gitt på emnesidene
(Legges ut 2. september).
Innlevering skjer ved opplasting til github. Mer informasjon om
klasseromsløsningen vi bruker på github vil bli gitt i god tid før
innleveringsfristen.

Resurser som er nyttige for oppgaven blir lagt ut på emnesiden,
under linken Obligatoriske oppgaver.

\section*{Del 1: Bash}

\subsection*{Filtre}
Før du begynner, må du pakke ut filtreet:
\begin{Verbatim}[fontsize=\small, frame=single]
$ tar -xxvf file_tree.tar.gz
\end{Verbatim}
\subsubsection*{Obs}
Første versjonen av obligen, var dette gjort med et skript.
Dette er endret.

\subsection*{Nyttige verktøy}
Alle komponenter brukt i oppgavene her må være en del av enten Bash
eller standard GNU/Linux-biblioteket.
Til hjelp nevner vi følgende tilgjengelige verktøy:

\begin{tabular}{rl}
    \verb;cat;      &   Concatenate files and print on the standard output. \\
    \verb;du;       &   Estimate file space usage. \\
    \verb;echo;     &   Display a line of text. \\
    \verb;find;     &   Search file tree for files. \\
    \verb;grep;     &   Print lines matching a pattern. \\
    \verb;rm;       &   Remove files or directories.\\
    \verb;sort;     &   Sort lines of text files. \\
    \verb;wc;       &   Print newline, word, and byte counts for each file. \\
    \verb;xargs;    &   Build and execute command lines from standard input.
\end{tabular}

Husk å gjøre skript eksekvebare før de kjøres:
\begin{Verbatim}[fontsize=\small, frame=single]
$ chmod +x skript.sh
\end{Verbatim}

\newpage

\subsection*{Oppgave 1.1}
List opp alle filer som er modifisert de siste \verb;n; dagene.
La utskriften inkludere og være sortet på filstørrelse.
For eksempel:
\begin{Verbatim}[fontsize=\small, frame=single]
$ ./list_new_files.sh filetree/ 80
36K     file_tree/Pkvye/htZiVgRE
48K     file_tree/Kq0Wv/5RYWI5kQ
104K    file_tree/Kq0Wv/MH/7GvTL2y
184K    file_tree/zg/grYxji7
564K    file_tree/zg/Hu/dNmOlK
644K    file_tree/Kq0Wv/MH/XhdhBbk
\end{Verbatim}
Merk at utskriften vil forandre seg med hvilken dag en kjøres.

\subsection*{Oppgave 1.2}
Find alle filer som inneholder et gitt ord.
Merk at vi snakker om på innsiden av filen, ikke filnavnet.
For eksempel, ved bruk av ordene ``what'' og ``hallo'' får vi
følgende:
\begin{Verbatim}[fontsize=\small, frame=single]
$ ./find_word.sh file_tree/ what
file_tree/Pkvye/vlfN/ZLbGhCmj:8687:NDa6gmZswhat77iTUFuoNiG23Y
$ ./find_word.sh filetree/ hello
No files containing "hello" found.
\end{Verbatim}

\subsection*{Oppgave 1.3}
Slett alle filer i filtreet med størrelse større enn en gitt verdi.
Størrelsen er gitt i kilobyte.
Print ut navnene på filene som slettes.
For eksempel:
\begin{Verbatim}[fontsize=\small, frame=single]
$ ./sized_delete.sh file_tree 750
Deleting...
file_tree/zg/Hu/vv/2KKnyIt5
file_tree/Pkvye/vlfN/ZLbGhCmj
file_tree/Kq0Wv/MH/zWG/8puxfjS
file_tree/Kq0Wv/MH/Z9kP8NB
$ ./sized_delete.sh file_tree 750
No files of size 750 kilobytes or larger found.
\end{Verbatim}

\subsubsection*{Hint}
Back-ticks tillater at man utfører en del-komando:
\begin{Verbatim}[fontsize=\small, frame=single]
count=`<bash command>`
\end{Verbatim}
Dette kan være nyttig å ha for å for eksempel identifisere om man
har 0 eller flere filer som passer størrelseskriteriet.

\subsection*{Oppgave 1.4}
Sorter linjene i en fil og lagre dem i en ny fil.
For eksempel:
\begin{Verbatim}[fontsize=\small, frame=single]
$ sh sort_file.sh unsorted_fruits sorted_fruits
$ cat sorted_fruits
apple
grape
orange
pear
pineapple
\end{Verbatim}

\section*{Del 2: Python}

I denne oppgaven skal dere skrive et Python skript som skaper et
filtre ikke veldig ulikt vedlagt i tar-formatet.
Skriptet skal inneholde følgende egenskaper:
\begin{itemize}
    \item Skriptet skal kunne skrives fra Bash/Cmd som et eget
        program og skal kreve de følgende tre argumentene:
        \verb;target;, \verb;folders; and \verb;files;.
    \item Det skal også være mulig å inkludere de følgende
        argumentene:
        \verb;size;, \verb;rec_depth;, \verb;start;, \verb;end;,
        \verb;seed; and \verb;verbose;
    \item \verb;target; bestemmer mappen hvor treet skal bygges.
    \item \verb;rec_depth; bestemmer nivået undermappene får lov til
        å gå.
    \item Hvis \verb;verbose; er inkludert (med hvilken som helst
        verdi), skal skriptet skrive ut de stegene den tar.
    \item Skriptet skal være tilfeldig (random) i de fleste
        aspekter:
        tilfeldig fil- og mappenavn, tilfeldig filinnhold,
        tilfeldig aksesert og modifisert tidstempling (atime \&
        mtime), og tilfeldig filstørrelse.
    \item Hvis man inkluderer \verb;seed; som en numerisk verdi,
        skal skriptet være deterministisk.
        Mao. skriptet skal gi samme resultat hvis repetert.
        Dette kan oppnås ved å bruke \verb;random.seed; funksjonen
        og kun hente tilfeldige elementer fra \verb;random;
        modulen.
    \item Argumentene \verb;dirs;, \verb;files;, \verb;size;,
        \verb;start; og \verb;end; er alle grenseverdier.
        Hver mappe skal ha $d\in$\verb;[0,dirs]; undermapper
        (gitt at \verb;rec_depth; ikke er nådd), og
        $f\in$\verb;[0,files]; filer.
        Hver fil skal inneholde $s\in$\verb;[1,size*1024]; karakterer
    \item Markørene for både aksesert tid (atime) og modifisert tid
        (mtime) skal ha $t\in$\verb;start,end;.
        Atime og mtime skal som hovedregel være ulike.
        Datoen skal for enkelthetens skyld være formatert i
        Unix/Epoch tid notert i sekunder.
\end{itemize}

Denne oppgaven skal løses platformsuavhengig.
Med andre ord, vil vi ikke akseptere løsninger som tilkaller Bash,
som \verb;os.system; eller \verb;Popen;.

Som hjelp kan du bruke malen: \verb;my_generate_file_tree.py;.
Modifiser den etter behov, men husk å endre dokumentasjon deretter.

\end{document}
