% Warning: Editing this file directly can cause erratic behaviour in the compiler.
\documentclass{article}

\usepackage[T1]{fontenc}
\usepackage[utf8]{inputenc}
\usepackage{fancyvrb}
\usepackage{framed}
\usepackage{color}
\providecommand{\shadedwbar}{}
\definecolor{shadecolor}{rgb}{0.87843, 0.95686, 1.0}
\renewenvironment{shadedwbar}{
\def\FrameCommand{\color[rgb]{0.7,     0.95686, 1}\vrule width 1mm\normalcolor\colorbox{shadecolor}}\FrameRule0.6pt
\MakeFramed {\advance\hsize-2mm\FrameRestore}\vskip3mm}{\vskip0mm\endMakeFramed}
\providecommand{\shadedquoteBlueBar}{}
\renewenvironment{shadedquoteBlueBar}[1][]{
\bgroup\rmfamily
\fboxsep=0mm\relax
\begin{shadedwbar}
\list{}{\parsep=-2mm\parskip=0mm\topsep=0pt\leftmargin=2mm
\rightmargin=2\leftmargin\leftmargin=4pt\relax}
\item\relax}
{\endlist\end{shadedwbar}\egroup}

\definecolor{codegray}{gray}{0.95}
\newcommand{\code}[1]{\colorbox{codegray}{\texttt{#1}}}

\newcommand{\bs}[1]{{\texttt{\textbackslash #1}}}

\title{Obligatorisk innlevering 3 høsten 2014, INF3331}
\author{Nicolai Skogheim <nicolai.skogheim@gmail.com>}
\date{\today}

\begin{document}
\maketitle

\tableofcontents

\section*{Oversikt}
\addcontentsline{toc}{section}{Oversikt}
Programmene som brukes til bildeprossesering ligger i MyImgTool-mappen.
Tester ligger i tests-mappen.
I resources-mappen ligger filer hentet fra student-resources.
\linebreak
Denne filen, og tilhørende .tex, ligger i report-mappen,
men kjøreeksempler i denne rapporten er alle ment å kjøres
fra oblig03-mappen.
For enkle kjøreeksempler kan det anbefales å ta en titt på README.md
i oblig03-mappen.

Denne rapporten kan for øvrig genereres med følgende kommandoer.
Obs. Må kjøres fra oblig03/report hvor denne filen ligger.
% Wrapped inline execution from line 51
\begin{Verbatim}[numbers=none,frame=lines,label=\fbox{{\tiny Terminal}},fontsize=\fontsize{9pt}{9pt},
labelposition=topline,framesep=2.5mm,framerule=0.7pt]
> python ../../oblig02/PreTeX/prepro.py report.xtex report.tex
> python ../../oblig02/PreTeX/compile.py ./report.tex
\end{Verbatim}
\noindent


\section{Implementasjon i Python, Numpy og Weave}
Hoveddelen i denne oppgaven var denoisingen.
Dette er python-løsningen:

% Imorted from line 61
\begin{shadedquoteBlueBar}
\fontsize{9pt}{9pt}
\begin{Verbatim}
def denoise(data,h,w, kappa=0.1, iter=10):
    """
    Runs a denoise algorithm on a given arary.
    Returns denoised channel.
    """
    logging.info("Running denoise.")
    for _ in xrange(0,iter):
        for i in xrange(1, w-1):
            for j in xrange(1, h-1):
                data[j*w+i] += \
                              kappa*(data[(j-1)*w+i]
                                   + data[j*w+(i-1)]
                                   - 4*data[j * w + i]
                                   + data[j*w+(i+1)]
                                   + data[i+ w * (j+1)]
                                   )
    return data
\end{Verbatim}
\end{shadedquoteBlueBar}
\noindent

Og her er løsningen med weave:

% Imorted from line 65
\begin{shadedquoteBlueBar}
\fontsize{9pt}{9pt}
\begin{Verbatim}
def denoise(data,h,w, kappa=0.1, iter=10):
    """
    Runs a denoise algorithm on a given arary.
    Returns denoised channel.
    """
    logging.info("Running denoise.")
    data_new = data.copy()
    tmp = data.copy()
    in_vars = ["data","data_new","tmp","kappa", "iter", "h", "w"]
    code=r"""
        for (int round=0; round<iter; round++)
        {
            for (int i=1; i<h-1; i++)
            {
                for (int j=1; j<w-1; j++)
                {
                    data_new(i,j) = data(i,j) + kappa*(data(i-1,j)
                    +data(i,j-1) -4*data(i,j) +data(i,j+1)
                    +data(i+1,j));
                }
            }
            tmp = data;
            data = data_new;
            data_new = tmp;
        }
    """

    comp=weave.inline(code,
                      in_vars,
                      type_converters=weave.converters.blitz)
    return data
\end{Verbatim}
\end{shadedquoteBlueBar}
\noindent


\section{Bruk av profilering}
Profilering av disse, i tilleg til fargeversjonen som nevnes senere, kan kjøres slik:
% Wrapped inline execution from line 70
\begin{Verbatim}[numbers=none,frame=lines,label=\fbox{{\tiny Terminal}},fontsize=\fontsize{9pt}{9pt},
labelposition=topline,framesep=2.5mm,framerule=0.7pt]
> python MyImgTool/profiling.py
\end{Verbatim}
\noindent

Og resultatet vil se omtrent slik ut:

% Av en eller annen grunn ville ikke prepro gi tilbake output, så
% da for vi gjøre det manuelt.
% Wrapped inline execution from line 78
\begin{Verbatim}[numbers=none,frame=lines,label=\fbox{{\tiny Terminal}},fontsize=\fontsize{9pt}{9pt},
labelposition=topline,framesep=2.5mm,framerule=0.7pt]
> python MyImgTool/profiling.py

Wed Nov 19 02:23:46 2014    python_timing
Wed Nov 19 02:23:46 2014    weave_timing
Wed Nov 19 02:23:48 2014    color_timing

         191170 function calls (191084 primitive calls) in 3.728 seconds

   Ordered by: cumulative time
   List reduced from 302 to 20 due to restriction <20>

   ncalls  tottime  percall  cumtime  percall filename:lineno(function)
        1    0.002    0.002    2.148    2.148 denoise_colors.py:181(run)
        1    0.021    0.021    1.833    1.833 denoise_colors.py:30(rgb2hsi)
        1    0.095    0.095    1.782    1.782 function_base.py:1843(__call__)
   187501    1.678    0.000    1.678    0.000 denoise_colors.py:45(hfunk)
        3    0.000    0.000    1.531    0.510 <string>:1(<module>)
        1    0.000    0.000    1.531    1.531 denoise.py:46(run)
        1    1.476    1.476    1.476    1.476 denoise.py:28(denoise)
        1    0.192    0.192    0.198    0.198 denoise_colors.py:64(hsi2rgb)
        3    0.000    0.000    0.096    0.032 inline_tools.py:133(inline)
        3    0.073    0.024    0.073    0.024 {apply}
        2    0.000    0.000    0.071    0.036 denoise_colors.py:125(denoise)
        2    0.000    0.000    0.068    0.034 inline_tools.py:361(attempt_function_call)
        1    0.000    0.000    0.050    0.050 denoise_weave.py:59(run)
        3    0.000    0.000    0.035    0.012 Image.py:1615(save)
        1    0.000    0.000    0.029    0.029 denoise_weave.py:26(denoise)
        1    0.000    0.000    0.028    0.028 denoise.py:19(list2img)
       26    0.011    0.000    0.028    0.001 {numpy.core.multiarray.array}
        3    0.000    0.000    0.027    0.009 JpegImagePlugin.py:566(_save)
        3    0.000    0.000    0.027    0.009 ImageFile.py:449(_save)
        1    0.005    0.005    0.027    0.027 denoise.py:8(img2list)
\end{Verbatim}
\noindent


For lettere å kunne sammenligne hastigheten på de tre programmene
kan det være greit å greppe etter "run".

% Wrapped inline execution from line 117
\begin{Verbatim}[numbers=none,frame=lines,label=\fbox{{\tiny Terminal}},fontsize=\fontsize{9pt}{9pt},
labelposition=topline,framesep=2.5mm,framerule=0.7pt]
> python MyImgTool/profiling.py | grep run --color=always
\end{Verbatim}
\noindent

% Wrapped inline execution from line 121
\begin{Verbatim}[numbers=none,frame=lines,label=\fbox{{\tiny Terminal}},fontsize=\fontsize{9pt}{9pt},
labelposition=topline,framesep=2.5mm,framerule=0.7pt]
python MyImgTool/profiling.py | grep run


        1    0.002    0.002    2.107    2.107 denoise_colors.py:181(run)
        1    0.000    0.000    1.530    1.530 denoise.py:46(run)
        1    0.000    0.000    0.044    0.044 denoise_weave.py:59(run)
\end{Verbatim}
\noindent

Jeg har valgt å ikke bruke timeit-modulen da jeg ikke ser hva den
kan gi meg som ikke cProfile kan.

\section{Utvidelse til farger}
Jeg må klare opp en ting helt først her. Jeg misforstod oppgaven.
I stedet for å utvide weave-implementasjonen lagde jeg en egen
versjon i en egen fil for dette.
Som det kommer fram senere i dette dokumentet har ikke det hatt
noe å si for sluttbrukeren om man bruker frontenden.

I denne oppgaven har jeg brukt akkurat samme funksjonalitet som i
weave-versjonen av svarthvitt denoising. Her kunne jeg faktisk
ha importert funksjonen fra denoise\_weave.py.

Når det kommer til konvertering mellom fargerom har jeg valgt å
bruke numpy til noe den er veldig god på, nemlig å jobbe med
store lister med data. Numpy gjør operasjonene i C og Fortran
så det skal være vel så raskt.

Før jeg går litt nærmere inn på koden vil jeg nevne en svakhet
programmet har på nåværende tidspunkt.
Utregningen av H i HSI gjøres ved å kjøre hvert element i rgb-listen
gjennom en funksjon, som returnerer verdien for h.
Dette er utrolig tregt, og er det eneste som gjør at dette programmet
ikke kjører fortere enn denoising av svarthvitt i ren python.

Det som stort sett foregår i konverteringsprosessen er at jeg lager
en ny array ved å kopiere en som har riktig størrelse og form (her
er det i(ntensity) som brukes som mal for s(aturation)).
Så setter jeg et predikat, eller en maske som det heter i numpyverdenen,
og ved å indeksere på masken kan jeg tilegne listen, her s, verdier
for de elementene hvor predikatet stemmer.
% Wrapped inline code from line 162
\begin{shadedquoteBlueBar}
\fontsize{9pt}{9pt}
\begin{Verbatim}
s = np.copy(i)
  m = i == 0
  s[m] = 0
\end{Verbatim}
\end{shadedquoteBlueBar}
\noindent

\section{Lineær manipulering}
Her er det ikke så mye å si, annet enn at det kan være lurt
å kjøre programmene med h-flagget for å se hvordan du setter verdier.
For eksempel brukes -u for hue siden h-flagget var tatt for help.
De andre kanalene bruker flagg man ville ha gjettet, som r for rød, s for saturation osv.

En kjøring kan gjøres direkte på denoise\_colors.py, eller gjennom
frontend som vist her:
% Wrapped inline execution from line 176
\begin{Verbatim}[numbers=none,frame=lines,label=\fbox{{\tiny Terminal}},fontsize=\fontsize{9pt}{9pt},
labelposition=topline,framesep=2.5mm,framerule=0.7pt]
> python MyImgTool/frontend.py -r 30 -u -10
\end{Verbatim}
\noindent

Kjører man med feil verdier vil programmet avbryte.
% Wrapped inline execution from line 181
\begin{Verbatim}[numbers=none,frame=lines,label=\fbox{{\tiny Terminal}},fontsize=\fontsize{9pt}{9pt},
labelposition=topline,framesep=2.5mm,framerule=0.7pt]
> python MyImgTool/frontend.py -r 30 -s -10

The saturation channel can be adjusted up or down by maximum 1
Your value was -10
\end{Verbatim}
\noindent

Legg til d-flagget for å kjøre denoise.

\section{Frontend}
Alene støtter de tre implementasjonene alle flaggene med unntak av
profilering, som jeg har valgt å legge i en egen fil (profiling.py).

Gjennom frontenden derimot, er alt mulig i henhold til oppgaven.
Som sagt så har jeg sett bort i fra timeit-modulen, men man kan
fortsatt sende med et t-flagg. Da vil cProfile kjøre og man
kan teste programmet med de parametere man ønsker.

Ved å kjøre frontend med h-flagget vil man få informasjon om alle muligheter.

Jeg har valgt å se bort fra eps.

\section{Testing og dokumentasjon}
Alle funksjoner er dokumentert.
Når det kommer til tester har jeg dessverre ikke levert på det nivået som
var forventet. Det er derimot lagt stor vekt på at alt av funksjonalitet
skal være lett å teste.
Det lille jeg har av tester viser hvor lett det kan være.
Law of Demeter har, og vil alltid være, en viktig rettesnor i så måte.

Testene kan kjøres med pytest, fra oblig03-mappa, på følgende måte:
% Wrapped inline execution from line 212
\begin{Verbatim}[numbers=none,frame=lines,label=\fbox{{\tiny Terminal}},fontsize=\fontsize{9pt}{9pt},
labelposition=topline,framesep=2.5mm,framerule=0.7pt]
> py.test
========================== test session starts ==========================
platform darwin -- Python 2.7.5 -- py-1.4.25 -- pytest-2.6.3
plugins: cov
collected 4 items

tests/denoise_color_test.py ....

======================== 4 passed in 0.25 seconds =======================
\end{Verbatim}
\noindent

Eller med nosetests om ønskelig:
% Wrapped inline execution from line 225
\begin{Verbatim}[numbers=none,frame=lines,label=\fbox{{\tiny Terminal}},fontsize=\fontsize{9pt}{9pt},
labelposition=topline,framesep=2.5mm,framerule=0.7pt]
> nosetests
....
----------------------------------------------------------------------
Ran 4 tests in 0.157s

OK
\end{Verbatim}
\noindent

Hva gjelder doctesting har vurdert det dithen at det ikke er noe av funksjonaliteten
i programmet som er hensiktsmessig å teste på denne måten.

\section*{Epilog}
\addcontentsline{toc}{section}{Epilog}
Det er mye som kunne vært skrevet om hvordan ting er gjort,
men som vanlig stoler jeg på at programmet er oversiktlig nok
til at leser uansett vil få mer utbytte av å gå rett til kilde(kode)n.
Det vil alltid være mitt fremste mål at koden er så selvdokumenterende
som det lar seg gjøre. Tester er en del av denne dokumentasjonen, så det
var dumt at jeg ikke fikk til å gjøre den delen bedre.


\end{document}

%PreTex data. Ignore following line.
%[1:1,2:2,3:3,4:4,5:5,6:6,7:7,8:8,9:9,10:10,11:11,12:12,13:13,14:14,15:15,16:16,17:17,18:18,19:19,20:20,21:21,22:22,23:23,24:24,25:25,26:26,27:27,28:28,29:29,30:30,31:31,32:32,33:33,34:34,35:35,36:36,37:37,38:38,39:39,40:40,41:41,42:42,43:43,44:44,45:45,46:46,47:47,48:48,49:49,50:50,54:54,55:55,56:56,57:57,58:58,59:59,60:60,54:58,55:58,56:59,57:60,58:61,59:62,60:63,62:88,63:89,64:90,66:129,67:130,68:131,69:132,72:72,73:73,74:74,75:75,76:76,77:77,72:139,73:139,74:140,75:141,76:142,77:143,111:111,112:112,113:113,114:114,115:115,116:116,117:117,118:118,119:119,120:120,121:121,122:122,123:123,124:124,125:125,126:126,127:127,128:128,129:129,130:130,131:131,132:132,133:133,134:134,135:135,136:136,137:137,138:138,139:139,140:140,141:141,142:142,143:143,144:144,145:145,146:146,147:147,111:181,112:181,113:182,114:183,115:184,116:185,119:119,120:120,121:121,122:122,123:123,124:124,119:192,120:192,128:128,129:129,130:130,131:131,132:132,133:133,134:134,135:135,136:136,137:137,138:138,128:204,129:204,130:205,131:206,132:207,133:208,134:209,135:210,136:211,137:212,138:213,139:214,140:215,141:216,142:217,143:218,144:219,145:220,146:221,147:222,148:223,149:224,150:225,151:226,152:227,153:228,154:229,155:230,156:231,157:232,158:233,159:234,160:235,161:236,166:166,167:167,168:168,169:169,170:170,171:171,172:172,173:173,174:174,175:175,166:247,167:247,168:248,169:249,170:250,171:251,172:252,173:253,174:254,175:255,178:178,179:179,180:180,181:181,182:182,183:183,178:262,179:262,180:263,186:186,187:187,188:188,189:189,190:190,191:191,192:192,193:193,194:194,186:273,187:273,188:274,189:275,190:276,191:277,192:278,193:279,194:280,195:281,196:282,197:283,198:284,199:285,200:286,201:287,202:288,203:289,204:290,205:291,206:292,207:293,208:294,209:295,210:296,211:297,222:222,223:223,224:224,225:225,226:226,227:227,228:228,229:229,230:230,231:231,232:232,233:233,234:234,235:235,222:312,223:312,224:313,232:232,233:233,234:234,235:235,236:236,237:237,238:238,239:239,240:240,241:241,242:242,232:325,233:325,234:326,235:327,236:328,237:329,238:330,239:331,240:332,241:333,242:334,243:335,244:336,245:337,246:338,247:339,248:340]